\title{%
  Towards Perfectly Secure and Deniable Communication Using an NFC-Based 
  Key-Exchange Scheme
}
\author{%
  Daniel Bosk\inst{1}
  \and
  Martin Kjellqvist\inst{2}
  \and
  Sonja Buchegger\inst{1}
}
\institute{%
  School of Computer Science and Communication,\\
  KTH Royal Institute of Technology, SE-100\,44 Stockholm\\
  Email: \email{\{dbosk,buc\}@kth.se}
  \and
  Department of Information and Communication Systems,\\
  Mid Sweden University, SE-851\,70 Sundsvall\\
  Email: \email{martin.kjellqvist@miun.se}
}

\maketitle
\begin{abstract}
  The increased electronic communication of today has many advantages for the 
  users.
  But it also comes with a cost: reduction in privacy.
  There are many tools in use to increase privacy in the electronic setting, 
  e.g.~PGP and Off-the-Record.
  While providing some desirable privacy properties, these protocols get other 
  unexpected properties in the modern surveillance state.
  When one actor can collect, store and search Internet-wide transcripts of 
  communication, some of the usually desirable properties force us to commit to 
  our communication.
  Thus we lose the possibility of deniability.
  In this paper we first analyse the possibility for deniability under a strong 
  adversary, who has an Internet-wide transcript of the communication.
  I.e.~we assume a stronger adversary model.
  Secondly, we present a scheme which provides the desirable properties of 
  previous schemes, but with stronger deniability under the new adversary 
  model.
  Our scheme requires physical meetings for exchanges of large amounts of 
  random key-material via near-field communication and later uses this random 
  data to key a one-time pad for text-messaging.
  % XXX Add results to the abstract
  Finally we evaluate the practical feasibility of the suggested scheme.

  \keywords{%
    Deniability,
    Perfect Secrecy,
    Surveillance,
    Off-the-Record,
    Key-Exchange,
    Near-Field Communication
  }
\end{abstract}

\acresetall{}
\section{Introduction}

We have learned a lot about modern government surveillance from the Snowden 
revelations starting in 2013.
For our current treatment, the most interesting ones are the tapping of 
fibre-optic cables~\cite{fibretap}, the storage of all intercepted encrypted 
data~\cite{cryptostore}, the search~\cite{xkeyscore} and visualization 
capabilities~\cite{boundlessinformant} for all intercepted data.
It is not the details that are interesting, it is the fact that one actor can 
collect, store and search Internet-wide transcripts of communication.
This paper focuses on the possibility of deniability in this setting.

Today, \ac{GPG}~\cite{gpg}, \ac{OTR}~\cite{otr} and Text\-Secure 
\cite{textsecure} are among the popular services used for private 
communication.
\ac{GPG} provides standard asymmetric and symmetric encryption, intended for 
use with email.
In 2004, \citeauthor{otr2004}~\cite{otr2004} first described the \ac{OTR} due 
to limitations of deniability in \ac{GPG}.
The design goal of the protocol is to achieve strong privacy properties for 
users' online communication, the same properties as expected from 
a face-to-face conversation.
The main application at the time was \ac{IM}.
In 2010, OpenWhisperSystems adapted the \ac{OTR} messaging 
protocol~\cite{frosch2014secure} for use in the smartphone text-messaging app 
TextSecure.

The construction used for deniability in \ac{OTR}, and the derived protocols, 
is based on the principle <innocent until proven otherwise>.
While this holds true for most civil societies, it is not true everywhere.
There are circumstances in which the principle <guilty until proven otherwise> 
is applied instead.
For these circumstances, with an adversary that can record all network traffic, 
it is not possible to create any false witness (proof-of-innocence) due to the 
deterministic nature of the protocol.
In Sect.~\ref{sec:undeniability} we show that this allows an adversary with 
transcripts of all network traffic to verify any statements about the 
conversation using the transcripts.
To thwart this we need truly deniable encryption, as defined by 
\citet{deniablecrypt}, which means that we need to introduce some randomness.

\subsection{Our Contributions}

We start from protocols like \ac{OTR}, but we assume a stronger adversary.
This stronger adversary model breaks some assumptions in \ac{OTR}-like 
protocols and removes the possibility for deniability.
We still want to achieve the same basic properties, e.g.~mutual authentication, 
but we also want to have stronger deniability.
Section~\ref{sec:ModelsOverview} gives a more detailed overview.

In Sect.~\ref{sec:undeniability} we show that an adversary who can record all 
communication in a network can use the deterministic properties of commonly 
used mechanisms to reject lies about any communication.
In Sect.~\ref{sec:deniability} we outline what is needed to achieve 
deniability.

Our main contribution, however, is presented in Sect.~\ref{sec:otp-kx}.
We use \ac{NFC} in smartphones to exchange large-enough amounts of random data 
when two users physically meet.
Later, when the users are apart, this data is used to key a \ac{OTP} for use 
when communicating.
This is a work-around to achieve deniable communication.
As we know from \citet{ShannonSecrecy}, if we use the \ac{OTP} correctly, then 
our communication is even perfectly secure.
Our scheme is described in detail in Sect.~\ref{sec:otp-kx}.

To estimate the feasibility of this scheme we investigate
\begin{itemize}
  \item the order of magnitude of random data needed to be able to cover 
    everyday text-message conversation;

  \item if the \ac{NFC} transmission rates and the random number generation in 
    combination with the number of physical meetings can provide high enough 
    exchange rates in practice; and

  \item how the continuous key-generation in this scheme affects battery life 
    in the device.
\end{itemize}

We answer the first question by estimating the amount of private communication 
for some users.
The details can be found in Sect.~\ref{sec:NeededRandomness}.

We answer the second question by estimating the required number of physical 
meetings for the same users.
Since we have an estimate of the amount of exchanged randomness needed and the 
transmission rates for the \ac{NFC} protocol, we can estimate how many physical 
meetings and how long transfers are needed to cover the needs.
The details can be found in Sect.~\ref{sec:Meetings}.

For the third question, we estimate the battery usage by performing key 
generation and exchanges using different Android-based phones while monitoring 
the battery consumption.
The details can be found in Sect.~\ref{sec:Battery}.

Finally we present our conclusions and suggest future work in 
Sect.~\ref{sec:Conclusions}.


\section{The System and Adversary Models}
\label{sec:ModelsOverview}

In this section we present the system and adversary model.
We start by presenting the system model.
Then we continue by presenting the adversary model and highlighting the 
differences compared to that of related protocols.

In our system model, we assume that each user has an \ac{NFC}-enabled 
smartphone.
This means that we have two communication channels: an \ac{NFC} channel and 
a public network-channel.
We will call them the private and public channel, respectively.
We can always use the public channel, but we can only use the private 
(\ac{NFC}) channel if we are in the same physical space.

We also assume that a user can, at least once, physically meet the party with 
whom they would like to communicate.
They must meet once before any electronic communication can be started.
In each meeting they exchange key-material.
We assume that a smartphone can generate cryptographically strong random data 
and that this random data is used as key-material.
Further, we assume that this key-material can be securely stored in the device.

For the adversary model, we assume a stronger adversary, Eve.
Eve records all traffic in the public channel.
This means that she records all traffic for the entire Internet.
Thus Eve has a transcript of all communication that has taken place in the 
public channel and any future communication will also be entered into her 
transcript.
But Eve cannot record any communication in the private channel, since she is 
assumed to be in a different physical space.
Also, Eve cannot access the devices used in the communication.
Instead, she will force us to reveal the keys to verify the \acp{MAC} and 
decrypt the ciphertexts she has recorded.

% XXX Treat that Eve controls the public channel
%We further assume that Eve is active, and that she controls the public 
%channel.
%I.e.~she might delay, modify or remove traffic there.

In summary, in related works, Eve has had the role of a prosecutor who must 
prove things to a judge.
In our model, Eve has the role of being both prosecutor and judge, which is 
more the case in some surveillance states.

\subsection{A Formal Definition of the Adversary}

We now proceed to give a formal definition of Eve, as she was described above.
She has the transcript of all communication in the public channel and she 
forces Alice to reveal the keys used for the ciphertexts in the transcribed
conversation with Bob.
Eve's task is to decide whether Alice is trying to lie.
We formalize this by the following game.

\begin{definition}[Surveillance-State Game]\label{def:StateGame}
  Let \(A\) be an efficient adversary and \(T = ( t_1, \ldots, t_n )\) be 
  a sequence of transcribed transmissions.
  Let \(\phi\) be the challenger algorithm and \(P = ( (m_1, k_1), \ldots, 
    (m_n, k_n) )\) the messages and keys corresponding to the transcripts in 
  \(T\).
  First let the adversary choose (the index of) a transcript \[
    i\gets A( r, T ),
  \] where \(r\rgets R\) is random coins sampled from a set \(R\).
  We create two challenges, \(c_0\) and \(c_1\):
  \begin{align*}
    c_0 &\gets k_i, \\
    c_1 &\gets \phi( r^\prime, i, P ),
  \end{align*}
  where \(r^\prime\rgets R\) is again some random coins.
  Next, we choose a bit \(b\rgets \{0, 1\}\) uniformly randomly.
  Finally, the adversary outputs a bit \[
    b^\prime\gets A( r, T, c_b ).
  \]
  We define the surveillance-state adversary's advantage as
  \begin{equation}
    2 \Pr[ b = b^\prime ] - 1.
  \end{equation}
\end{definition}

Alice uses the challenger algorithm \(\phi\) to produce a key \(k_i^\prime\neq 
  k_i\).
The goal of our scheme is that the Eve cannot distinguish the two keys using 
the transcript \(t_i\).
Next we will outline the problems other protocols have against this adversary.
We will use \ac{OTR} as an example, but similar arguments can be made against 
similar protocols.
Then we will return to our solution in Sect.~\ref{sec:deniability} and onwards.


\section{Why Alice and Bob Currently Must Forget Their Conversation}
\label{sec:undeniability}

The security of today's popular services --- \ac{GPG}, \ac{OTR} and TextSecure 
--- rely on the standard cryptographic mechanisms.
These mechanisms provide strong security properties.
In this section we outline why some of these properties are too strong for 
deniability in the setting where the adversary has a transcript of all 
communications.

\ac{GPG} provides asymmetric and symmetric encryption intended to be used with 
email.
\citet{otr2004} have already presented arguments against \ac{GPG} (and 
\acl{PGP}, \acs{PGP}), but we will summarize them here.
If Alice wants to send a message \(m\) to Bob, then she will encrypt it for 
Bob's public key \(k_B\).
She will then create a signature for the resultant ciphertext \(c = \Enc_{k_B}( 
m )\) with her own private key \(k_A^{-1}\), i.e.~\(s = \Sign_{k_A^{-1}}( H(c) 
)\).
Alice will then send the ciphertext block and the signature to Bob, and this 
transaction will be recorded in Eve's transcript.
This scheme provides non-repudiation, i.e.~Alice can not deny having sent the 
message \(m\) at a later time and Bob can also prove to a third party that 
Alice sent \(m\).
Further, Eve can also prove that Alice sent \(c\), but she can only verify the 
plaintext \(m\) if Bob would reveal it to her.

In their paper, \citet{otr2004} suggested a scheme which does not have this 
problem: the \ac{OTR} messaging protocol.
This protocol provides authentication for Alice and Bob, so that they can trust 
they are talking to the right person.
But they can do no more than that, Bob can no longer prove to a third party 
what Alice has sent.
They accomplish this by a continuous use of the \ac{DH} key-exchange and 
a \ac{MAC} based on symmetric keys.
We provide a simplified description here, mainly to give an understanding of 
underlying ideas, see the original paper~\cite{otr2004} for a detailed 
description.
Alice chooses a secret exponent \(a\) and Bob chooses a secret exponent \(b\).
Alice signs \(g^a\) and sends \[
  A\to B\colon g^a, \Sign_{k_A^{-1}}( g^a )
\] to Bob.
Bob conversely sends \[%
  B\to A\colon g^b, \Sign_{k_B^{-1}}( g^b )
\] to Alice.
By this time they can both compute the secret shared-key \(k = g^{ab}\).
Let \(H_E\) and \(H_M\) be two cryptographically secure hash functions, used 
for deriving encryption and \ac{MAC} keys, respectively.
When Alice wants to send the message \(m\) to Bob, she chooses a random 
\(a^\prime\) and sends \[
  A\to B\colon g^{a^\prime}, c = \Enc_{H_E( k )}( i )\oplus m,
  \MAC_{H_M( k )}( g^{a^\prime}, c ),
\] where \(i\) is some counter, to Bob.
Once she knows Bob has received the message she also sends the \ac{MAC} key 
\(H_M( k )\) to Bob.
The next time Alice wants to send a message to Bob, she will use \(k^\prime 
= g^{a^\prime b}\).

Now, Bob can no longer prove to a third party what Alice has said.
This is due to the \ac{MAC} being based on a secret key which Bob has access 
to.
Also, since the encryption is done in counter mode~\cite{blockmodes}, the 
ciphertext is malleable.
This means that flipping a bit in the ciphertext, yields the same flip in the 
plaintext.
Thus, anyone possessing the \ac{MAC} key can modify the plaintext by flipping 
the bits in the ciphertext and then generate a new \ac{MAC}.

\subsection{Verifying Who Sent What}

The arguments for forgeability using malleable encryption and publishing the 
\ac{MAC} keys only hold if the adversary cannot trust the source of the 
transcript.
This more powerful Eve can ultimately trust the transcript since she collected 
it herself from the network.
And if the courts trust Eve, they also trust the transcript.

In this setting the forgeability property vanishes.
Eve knows that no one has modified the ciphertext, she recorded in her 
transcript as it left Alice and arrived to Bob.
She also recorded Alice publishing the \ac{MAC} key used for the signature.
This allows Eve to use the \ac{MAC} for each ciphertext to verify them.
She knows that Alice is the author of a message because she observes when Alice 
publishes the \ac{MAC} key.
Thus, Eve also knows that no one has used the malleability property, because if 
they did, that action would be recorded in Eve's transcript.

\subsection{Verifying Encryption Keys}

Furthermore, Eve also learns some information about the key from the ciphertext 
and \ac{MAC}.
Eve can use the \ac{MAC} to discard false keys for the ciphertext.
Since Eve has \(s = \MAC_{H_M( k )}( c )\) for a ciphertext \(c\) recorded in 
her transcript, she can reject a key \(k^\prime\neq k\) by verifying that
\(\MAC_{H_M( k^\prime )}( c ) \neq s\).
Hence, by having the \ac{MAC} key depend on the encryption key, we 
automatically decrease the number of spurious keys and thus also reduce our 
possibility for deniability.

\subsection{How Hard Is Deniability?}
\label{sec:HardnessOfDeniability}

As suggested above, we have difficulty achieving deniability.
This is illustrated by the following equations.
Assume
\begin{equation*}
  \Enc_{H_E(k)}( m ) = c = \Enc_{H_E(k^\prime)}( m^\prime )
\end{equation*}
and \(k\neq k^\prime\), then
\begin{equation*}
  \Pr\left[
    \MAC_{H_M(k)}( c ) = \MAC_{H_M(k^\prime)}( c )
  \right]
  \approx
  \Pr\left[ H_M(k) = H_M(k^\prime) \right].
\end{equation*}
I.e.~our chance of lying about the key \(k\), replacing it with a key 
\(k^\prime\), is reduced to finding a collision for the hash function \(H_M\).
(There is also the negligible probability of \(\MAC_x(c) = \MAC_{x^\prime}(c)\) 
for \(x\neq x^\prime\) to consider.)

Furthermore, we find the key \(k^\prime\) by finding the preimage of \(H_E( 
k^\prime )\).
And if the encryption system \(\Enc\) is a trap-door permutation, then we will 
have to break that first, just to find \(H_E( k^\prime )\) before we can 
attempt finding its preimage.


\section{Requirements for Deniability}
\label{sec:deniability}

To be able to get deniability in our given scenario, Alice and Bob need to be 
able to modify the plaintext without modifying the ciphertext.
They also need a \ac{MAC} key independent from the encryption key.
Then they can change the encryption key and the plaintext, but the ciphertext 
and \ac{MAC} remains the same.

\citeauthor{deniablecrypt} gave the original formal definition of deniable 
encryption in their seminal paper~\cite{deniablecrypt}.
We will give their definition of sender-deniable encryption for shared-key 
schemes here.
\begin{definition}[Shared-key sender-deniable 
  encryption]\label{def:DeniableEnc}
  A protocol \(\pi\) with sender \(S\) and receiver \(R\), and with security 
  parameter \(n\), is a shared-key sender-deniable encryption protocol if:
  \begin{description}
    \item[Correctness] The probability that \(R\)'s output is different than 
      \(S\)'s output is negligible (as a function of \(n\)).

    \item[Security] For any \(m_1, m_2\in M\) in the message-space \(M\) and 
      a shared-key \(k\in K\) chosen at random from the key-space \(K\), then 
      we have \(\Pr[ \Enc_k( m_1 ) = c ] \approx \Pr[ \Enc_{k^\prime}( m_2 
      ) = c ]\).

    \item[Deniability] There exists an efficient <faking> algorithm \(\phi\) 
      having the following property with respect to any \(m_1, m_2\in M\).
      Let \(k, r_S, r_R\) be uniformly chosen shared-key and random inputs of 
      \(S\) and \(R\), respectively, let \(c = \Enc_{k, r_S, r_R}( m_1 )\) and 
      let \((k^\prime, r_S^\prime) = \phi( m_1, k, r_S, c, m_2 )\).
      Then the random variables \[
        ( m_2, k^\prime, r_S^\prime, c ) \text{ and }
        ( m_2, k, r_S, \Enc_{k, r_S, r_R}( m_2 ) )
      \] are distinguishable with negligible probability in the security 
      parameter \(n\).
  \end{description}
\end{definition}
This means that given a ciphertext \(c = \Enc_k( m )\) and a false plaintext 
\(m^\prime\), there exists a polynomial-time algorithm \(\phi\) such that 
\(\phi( c, m^\prime ) = k^\prime\) yields a key \(k^\prime\) and \(m^\prime 
= \Dec_{k^\prime}( c )\).
As we illustrate in Sect.~\ref{sec:HardnessOfDeniability}, there exists no such 
polynomial-time algorithm \(\phi\) for \ac{OTR} or \ac{GPG}.

One encryption system for which the algorithm \(\phi\) is trivial is the 
\ac{OTP}.
\begin{definition}[One-Time Pad]\label{def:OTP}
  Let \(M = K = {(\Z_2)}^n\).
  Then let \(m\in M\) be a message in the message-space \(M\), let \(k\in K\) 
  be a uniformly chosen key in the key-space \(K\).
  Then we define \[
    \Enc_k( m ) = m\oplus k \text{ and } \Dec_k = \Enc_k.
  \]
\end{definition}
\citet{ShannonSecrecy} proved that this scheme is perfectly secret.
But this requires that the key \(k\) is as long as the message \(m\).
The key must be uniformly chosen, i.e.~never reused.
This is why this scheme is usually considered impractical.

We can easily see, and it is also pointed out in~\cite{deniablecrypt}, that the 
\ac{OTP} fulfils Def.~\ref{def:DeniableEnc}.
We can simply define \(\phi( m_2, c ) = m_2\oplus c\) and this would yield 
\(k^\prime\) such that \[
  \Dec_{k^\prime}( c ) = c\oplus k^\prime = c\oplus ( m_2\oplus c ) = m_2.
\]

We also want to resolve the problem of the \ac{MAC} being a witness for the 
correct key.
The problem in the previous schemes is that the \ac{MAC} key is derived from 
the same master key as the encryption key.
Instead of deriving the encryption key and the \ac{MAC} key by using two 
different key-derivation functions on the same master key, we have to use 
information-theoretically independent keys.

We still want authenticated encryption though.
\citet{AuthEncJournal} proved that first encrypting the plaintext and then 
generating a \ac{MAC} for the ciphertext, always provides secure authenticated 
encryption.
Since this authenticates the ciphertext, and not the plaintext, it will not 
interfere with our deniability.
This way, when using the \ac{OTP} we can keep the \ac{MAC} key fixed while 
adapting the encryption key to our new plaintext, then hand the keys for both 
\ac{MAC} and encryption to the adversary as a (false) witness.
We summarize this in the following theorem.

% XXX Prove that a deniable scheme is still deniable after EtM
\begin{lemma}\label{lem:StillDeniable}
  The \ac{OTP} composed with encrypt-then-MAC provides integrity for the 
  ciphertext but not for the plaintext.
\end{lemma}
\begin{proof}
  We know from above (and~\cite{DeniableEncryption}) that the \ac{OTP} is 
  a shared-key deniable encryption scheme.
  \dots
\end{proof}

\begin{corollary}
  A shared-key deniable encryption scheme is still shared-key deniable after 
  being composed with encrypt-then-MAC authenticated encryption.
\end{corollary}

However, we also need to treat the fact that the \ac{OTP} is a stateful 
encryption scheme, where encryption and decryption are both stateful.

\begin{lemma}\label{lem:OTPIntegrity}
  Stateful encryption and decryption composed with encrypt-then-MAC provides 
  authenticated encryption.
\end{lemma}
\begin{proof}
  We know from \citet{AuthEncryption} that applying a strongly unforgeable 
  \ac{MAC} to a ciphertext yields secure authenticated encryption.
  \dots
\end{proof}


\section{Achieving Deniability}
\label{sec:otp-kx}
% XXX Move OTP specifics to sec:Implementation

Due to the deniability requirements outlined above, the randomness used for 
encryption cannot be extended by a \ac{PRNG}: if we do, then we are in the same 
situation as when we were using a trap-door permutation --- we cannot 
efficiently find a seed to the \ac{PRNG} which yields a stream that decrypts 
the ciphertext to the desired plaintext.
Instead we generate randomness continuously and then exchange it using the 
\ac{NFC} functionality of smartphones.
This way we can use the everyday chance-encounters for exchanging the generated 
randomness when we meet, and then use it to key the \ac{OTP} scheme when 
physically apart.

From a user perspective, putting two phones together <charges the deniable 
encryption tool>.
This is probably a good metaphor to build on, since it builds on the mental 
model of a battery.
Users are already familiar with this model, and thus, when running low on 
randomness, fewer messages should be exchanged until another physical meeting 
can be arranged to <charge> the tool again.

\subsection{The Protocol}
\label{sec:Protocol}

Alice and Bob want to communicate securely with the possibility of deniability 
in this adversary model.
Alice and Bob start by each generating a long string of random bits.
Alice generates the string \(k_A\) of length \(l_A\) from a source of 
randomness \(R_A\).
Bob conversely generates \(k_B\) or length \(l_B\) from a source of randomness 
\(R_B\).
When Alice and Bob meet, they exchange \(k_A\) and \(k_B\) over the private 
channel.
Thus Eve cannot see this traffic.

Alice and Bob part, and later Alice wants to send a message to Bob over the 
public channel.
For Alice to send the message \(m\), the message length \(|m|\) and the size of 
a \ac{MAC} key, say 256 bits, must be less than \(l_A\).
Alice then takes the first \(|m|\) bits from \(k_A\), let us denote those bits 
as \(k_{A,m}\), and computes \(c = m\oplus k_{A,m}\).
She then takes another 256 bits, say \(k_{A,\MAC}\), and computes a \ac{MAC} 
tag \(t = \MAC_{k_{A,\MAC}}( c )\).
She sends the ciphertext and \ac{MAC} tag, \((c, t)\), to Bob, and finally 
discards the used key-material.

Bob does the converse to send a message (back) to Alice, but he uses bits from 
the bit string \(k_B\) as key-material.
The protocol is illustrated in Fig.~\ref{fig:KeyExchange} and~\ref{fig:Comm}.

\begin{figure}
  \centering
  \begin{sequencediagram}
    \newinst{A}{Alice}
    \newinst[1]{B}{Bob}

    \mess{A}{}{B}
    \node[anchor=east] at (mess from) {$k_{A,1}, \ldots, k_{A,l}$};

    \mess{B}{}{A}
    \node[anchor=west] at (mess from) {$k_{B,1}, \ldots, k_{B,l^\prime}$};

  \end{sequencediagram}
  \caption{%
    A sequence diagram illustrating the key exchange.
    Eve cannot record this as it is communicated over the private channel.
  }\label{fig:KeyExchange}
\end{figure}

\begin{figure}
  \centering
  \begin{sequencediagram}
    \newinst{A}{Alice}
    \newinst[1]{E}{Eve}
    \newinst[1]{B}{Bob}

    \mess{A}{}{E}
    \node[anchor=east] at (mess from)
    {\shortstack{$\Enc_{k_{A,m}}( m ) = c$, \\ $\MAC_{k_{A,\MAC}}( c )$}};
    \prelevel{}
    \mess{E}{}{B}

    \mess{B}{}{E}
    \node[anchor=west] at (mess from)
    {\shortstack{$\Enc_{k_{B,m^\prime}}( m^\prime ) = c^\prime$, \\
    $\MAC_{k_{B,\MAC^\prime}}( c^\prime )$}};
    \prelevel{}
    \mess{E}{}{A}
  \end{sequencediagram}
  \caption{%
    A sequence diagram illustrating the correspondence.
    Eve records this information as it is sent over the public channel.
  }\label{fig:Comm}
\end{figure}

\subsection{The Security of the Protocol}

Now we cover the security of the protocol.
The properties we want to have in the protocol are deniable encryption with 
authenticated ciphertexts.

\begin{theorem}
   If the entropy of the keys exceeds the message lengths and the \ac{MAC} 
   scheme is strongly unforgeable, then the protocol in 
   Sect.~\ref{sec:Protocol} provides perfectly secure and deniable encryption 
   with authenticated ciphertexts.
\end{theorem}
\begin{proof}[sketch]
  \citet{ShannonSecrecy} showed that if the entropy of the key exceeds the 
  entropy of the message, then we have perfect secrecy.
  The protocol ensures that no key is reused, neither for encryption nor 
  \ac{MAC}.
  If we combine this with Lem.~\ref{lem:StillDeniable} 
  and~\ref{lem:OTPIntegrity}, then we have the desired result.
  \qed{}
\end{proof}

This means that we can have perfectly secure and deniable encryption if we can 
generate random data with high-enough entropy.

\subsection{Some Extensions}

A problem that can occur is that Alice and Bob might run out of key-material 
before they can meet again.
One way to handle this is for them to communicate less as they are closing in 
on the end of their random bit strings and use the last of the randomness to 
schedule a new meeting.

An alternative way they can handle this problem is to switch to another scheme, 
but with the knowledge that it is no longer deniable.
In a similar fashion, Alice and Bob might not need deniability for all their 
communications.
Thus they can switch to e.g.~\ac{OTR} or TextSecure when they do not need 
deniability against Eve, and then switch back when they want deniability.
This strategy would use less randomness and they need to meet less often.

Alice can do as in \ac{OTR} and publish the \ac{MAC} key when she receives 
a reply from Bob.
The effect we might get through this is that since the \ac{MAC} key is recorded 
in Eve's transcript, this might lower the trust in Eve's transcript.


\section{Implementation and Evaluation}

We have developed an app\footnote{%
  The source code is available at URL 
  \url{https://github.com/MKjellqvist/OTPNFCTransfer/}.
} for Android devices which implements the above ideas.
It generates randomness continuously in the background to build up a pool of 
randomness.
It can also exchange this randomness with another phone over \ac{NFC}.

To estimate the feasibility of this scheme, we investigate the amount of 
randomness needed and how much randomness we can generate 
(Sect.~\ref{sec:NeededRandomness}), the required and possible transfer times 
over \ac{NFC} (Sect.~\ref{sec:Meetings}), and finally how this scheme affects 
the battery consumption of the smartphone (Sect.~\ref{sec:Battery}).
The methodology and results are given in each respective section below.

\subsection{The Amount of Randomness Needed}
\label{sec:NeededRandomness}
Since we use the \ac{OTP}, we need as much key material for encryption as we 
have plaintext.
We need some additional key-material for the \acp{MAC}, e.g.~128--256 bits 
per sent message.
Thus we can estimate the total amount of randomness needed by estimating the 
exchange rate of plaintext.
To do this we analyse the Enron email dataset\footnote{%
  The source code for the data analysis described below is available at URL 
  \url{https://github.com/dbosk/mailstat/}.
}.

We are interested in personal communication, i.e.~we are not interested in 
newsletters and the like.
We are not interested in attachments either, so we discard those.
To filter out the newsletter category of messages, we rely on emails found in 
the users <sent> directory, since these are emails sent by real users.
Although we were not interested in newsletter-like emails, the sent directories 
might include forwarded newsletters.
However, we argue that these should be included since the user actively wanted 
to tell someone else about the content.

%Since we are using the \ac{OTP}, we also use key material for the replies.
%We thus also include the received replies to the sent emails.
%The rationale for this is that received replies are not necessarily from people 
%within the Enron company, but the emails are written by real users and should 
%thus be included to give us more accurate data.

Since this dataset contains a mix of corporate and private emails, and is 
fairly small, it is hard to draw any general conclusions from it.
So the Enron dataset is just one example.
Another dataset, communication using other media, e.g.~text messages rather 
than email, would probably change the user behaviour and these numbers.
But our main goal is to get an estimate of user communication to see whether 
our scheme is completely infeasible or not, and we argue that this dataset lets 
us reach that goal.

\begin{pycode}[random]
import math
import sqlite3
import pathlib
import sys
import decimal

sys.path.insert( 0, "mailstat" )
import mailstat
#import libsci

metadata = sqlite3.connect( "enron-sent.sqlite3" )
prectime = decimal.Decimal( "0.1" )
precdata = decimal.Decimal( "1000" )

mean_msg_size, stddev_msg_size = \
  mailstat.mean_message_size( metadata )
mean_msg_size = mean_msg_size.quantize( precdata )
stddev_msg_size = stddev_msg_size.quantize( precdata )

mean_msg_freq, stddev_msg_freq = ( decimal.Decimal(40), decimal.Decimal(5) )
#  mailstat.mean_message_frequency( metadata )
mean_msg_freq = mean_msg_freq.quantize( precdata )
stddev_msg_freq = stddev_msg_freq.quantize( precdata )

mean_contacts, stddev_contacts = \
  mailstat.mean_number_of_contacts( metadata )
mean_contacts = mean_contacts.quantize( precdata )
stddev_contacts = stddev_contacts.quantize( precdata )

data_per_day = ( mean_msg_size + stddev_msg_size ) * mean_msg_freq
\end{pycode}

\marginnote{\raggedright{}%
  \textbf{Note:}
  The numbers in this and the following sections are preliminary estimates from 
  the dataset.
  We have not yet analysed the dataset in detail to discover any unforeseen 
  problems.
  This is also the reason why they are rounded to have one or no decimals, 
  rather than rounded according to their number of significant digits.
  All numbers will be rounded to the number of significant digits in the final 
  version of the paper.
}
In the Enron dataset, we found that the average message was
\(\unit{\py[random]{mean_msg_size}}{\byte}\)
excluding any headers and attachments.
The standard deviation was
\(\unit{\py[random]{stddev_msg_size}}{\byte}\).
The large standard deviation can probably be explained by the data being 
emailed:
If a conversation requires a few rounds, then the previous messages accumulate 
in the body of the email as included history.

We also found that the average user communicates with
\(\py[random]{mean_contacts}\)
other users.
The standard deviation was
\(\py[random]{stddev_contacts}\).
This means that a user has approximately
\(\py[random]{(mean_contacts 
+ stddev_contacts).quantize(decimal.Decimal(10))}\)
users to exchange keys with.
\reversemarginpar{}
\marginnote{\raggedright{}%
  \textbf{Note:}
  We have not yet estimated the message frequency in the dataset.
  We chose the number \py[random]{mean_msg_freq} as a reasonably high guess.
  The estimated value from the dataset will be available in the final version 
  of the paper.
}
\reversemarginpar{}
If a user sends
\(\py[random]{mean_msg_freq}\)
messages per day, then we need on average less than
\(\unit{\py[random]{(data_per_day/1024).quantize( 10 )}}{\kibi\byte}\)
per day.
This means that we need approximately
\(\unit{\py[random]{(data_per_day*365/1024/1024).quantize( precdata 
)}}{\mebi\byte}\) to store one year's key-material.


We use Android's <SecureRandom> to generate our randomness.
This is the only supported way to generate randomness on the Android platform, 
and it allows us to generate enough amounts of random data.
Some research \cite{AndroidLowEntropyMyth,JavaRandomness} suggest that 
<SecureRandom> under certain circumstances uses a low entropy seed.
However, the documentation states that SecureRandom can be relied upon for 
cryptographic purposes.
Thus the security of SecureRandom for use with the \ac{OTP} must be 
investigated further.

\subsection{The Number of Meetings and Transfer Time}
\label{sec:Meetings}
From the above analysis, we know the average amount of data communicated 
between users per day.
We also know that the \ac{NFC} protocol can achieve a transmission rate of up 
to \unit{424}{\kilo\bit\per\second}~\cite{NFCController}.
In Tab.~\ref{tbl:MeetingsTradeoff} we tabulate how often two users meet 
compared to how much key-material they would need on average until the next 
meeting and how long time this data would take to transfer using \ac{NFC}.
The times provided does not include the setup of the \ac{NFC} radio channel, 
only actual transmission is considered. The setup phase takes about 
\unit{5}{\second} on the tested devices.

% XXX improve visualization of the data
% - Mitra had a good suggestion
\begin{table}
  \centering
  \caption{%
    The table illustrates how the frequency of the key-exchanges affects the 
    required time for each key-exchange.
    Less frequent exchanges requires larger exchanges and, hence, more time.
  }
  \label{tbl:MeetingsTradeoff}
  \begin{pycode}[random]
print( r"""
\begin{tabular}{lrr}
  Exchange &
  Time (\second) &
  Key-material (\kibi\byte) \\
  \toprule
""" )

timespans = {#
  1 : "Daily",
  7 : "Weekly",
  30 : "Monthly",
  60 : "Bimonthly",
  365 : "Annually"
}

for i in sorted( timespans.keys() ):
  needed_data = ( data_per_day * i ).quantize( precdata )
  needed_time = needed_data / ( decimal.Decimal( 424 ) * 1000 / 8 )
  needed_time = needed_time.quantize( prectime )
  print( r"%s & \(%s\) & \(%s\) \\" % \
  ( timespans[i],
    ( needed_time ).quantize( prectime ),
    ( needed_data/1024 ).quantize( precdata )
    ) )

print( r"""
  \bottomrule
\end{tabular}
""" )
  \end{pycode}
\end{table}

Considering this, we can see that a user needs to spend an order of 10s of 
seconds per day doing key-exchanges.
This number is divided among the contacts with whom the user communicates.
More frequently communicating contacts will require a larger part of the 
key-exchanges.

\subsection{The Battery Consumption}
\label{sec:Battery}
To estimate the effects on battery consumption we find a typical RF-active 
rating of \unit{60}{\milli\ampere} for the NFC chip~\cite{NFCController}.
The battery effects of this is negligible and on the order of 
\unit{2}{\text{\textperthousand}} of the battery charge at the considered 
usages.

To estimate the effects on battery consumption we first build a baseline.
For this we used the Android systems build-in power-consumption estimates.
We used one phone as a reference and two others running the app implementing 
our scheme.

For the component generating the randomness, tests were performed where we 
generated the annual demand of key-material.
This provided no indication of battery drain.
The processor load was measured at \unit{2}{\%} and the input-output load was 
  measured at \unit{15}{\%}. 


\section{Conclusions}
\label{sec:Conclusions}

We set out to design a scheme which provides users with deniability in 
a stronger adversary model.
Provided that we can generate random data with high-enough entropy, then our 
protocol provides
\begin{itemize}
  \item perfect secrecy,
  \item authenticated and
  \item deniable encryption.
\end{itemize}
However, to achieve this scheme and these properties, we require physical 
meetings to exchange the randomness.
If Alice and Bob run out of randomness they can fall back to e.g.~\ac{OTR}, but 
then they lose deniability against Eve.
In either case, they are never worse off than using \ac{OTR} or TextSecure.

We also showed that our scheme is usable.
We found that a typical exchange of key material requires less than 
\unit{10}{\second} daily to complete.
If you exchange the key-material on a weekly basis, then it is still less than 
a minute, while monthly and bimonthly require up to five minutes.
Thus the transmission rates are not a usability concern.
Also, the effects on battery life under the considered use is not a limiting 
factor in neither the generation of the key-material nor the transmission of 
the key-material.

The method for estimating the needed amount of data can be improved.
This estimate depends on the type of communication, e.g.~corporate emails 
differs from personal text-messaging.
To get more accurate estimates, it might better to evaluate a dataset from 
other settings.
To better estimate communication needs for private individuals, it might be 
better to use text-messages (SMSs).
However, we intended to show that our scheme is feasible, and we argue that we 
have reached that goal.

The only issues found in the scheme are related to the <if> regarding 
high-enough entropy data.
The security of SecureRandom for use with the \ac{OTP} must be investigated 
further.
In addition to~\cite{JavaRandomness,AndroidLowEntropyMyth}, we also have the 
result of \citet{UniversalityOTP} to consider.

As a final note, the design of the \ac{NFC} API is hindering the flexibility of 
ours and similar solutions.
We are mostly concerned about the following points:
\begin{itemize}
  \item There is no mechanism in which to stream data over \ac{NFC}\@.
    This is desirable from a usability standpoint of the app, in particular 
    with regards to interrupted transmissions.
    This might be solved by a more innovative implementation.
  \item The transmission must be done in the form of files, and currently these 
    have to reside on a publicly readable file-system in the device.
    This is a concern for both the confidentiality and integrity of the 
    key-material, as the transmitted files can be intercepted by a malicious 
    app competing for the received files.
\end{itemize}

\subsection{Future Work}

There are several interesting directions to follow from this work.
We start with the technical one.
The security of the actual \ac{NFC} transfer was out of the scope of this work.
However, the security of the \ac{NFC} protocol must be considered: in what 
proximity can Eve successfully record the \ac{NFC} traffic?
For instance, \citet{RFIDProximity} found that RFID tags could be read over 
\unit{50}{\metre} away in a hallway.
But more interestingly, what are the countermeasures?

Next, we can argue the need for deniability as compared to not being able to 
reveal any keys.
An interesting first step in this direction would be to conduct a study with 
users: what is the users' perception of deniability, what is more convincing?
This would also be interesting to contrast by looking into game theory to see 
what can be said about the behaviour of a probable liar: do we gain any 
credibility using this deniable scheme over simply refusing to disclose the 
keys?
What are the differences if we have a rational adversary compared to an 
irrational one?
Finally, there is the legal perspective, which could probably also benefit from 
exploring these questions.

Another direction, into usable security and privacy, would be to study suitable 
metaphors and mental models for this kind of system.
We suspect that the mental model of <charging deniability> when we exchange 
randomness is good, i.e.~that it does not lead to any contradictory behaviours 
which might put the user's security and privacy at risk.
Our guess is that this is more intuitive than e.g.~asymmetric encryption.

Finally, to support the user, it would be interesting to have a <budget> 
algorithm for the exchanges.
This algorithms would take into account the users' average communication rate 
and the users' average exchange frequencies, use these data to support the user 
in planning for future key-exchanges if the communication pattern changes or 
deviates from normal.


\subsubsection*{Acknowledgements}

This work was funded by the Swedish Foundation for Strategic Research grant SSF 
FFL09-0086 and the Swedish Research Council grant VR 2009-3793.
We would like to thank the anonymous reviewers for valuable feedback.


\printbibliography{}

